\documentclass[
  % anonymous,
  fontset = mac,
]{shtthesis}

\shtsetup{
  degree = {master},
  degree-name = {工学硕士},
  degree-name* = {Master~of~Science~in~Engineering},
  secret-level = {白给},
  title = {\ShtThesis~v\version~使用说明},
  title* = {A~User's~Guide\\to\\\ShtThesis~v\version},
  keywords = {上海科技大学,学位论文,\LaTeX{}},
  keywords* = {ShanghaiTech~University, Thesis, \LaTeX{}},
  author = {李润东},
  author* = {Li~Rundong},
  institution = {上海科技大学信息科学与技术学院},
  institution* = {School~of~Information~Science~and~Technology\\ShanghaiTech~University},
  supervisor = {范睿~副教授},
  supervisor* = {Professor~Fan~Rui},
  supervisor-institution = {上海科技大学信息科学与技术学院},
  discipline-level-1 = {计算机科学与技术},
  discipline-level-1* = {Computer~Science~and~Technology},
}

\usepackage{xcolor}
\definecolor{shtred}{RGB}{146,46,23}
% `latex' and `shell' environments are adapted from `thuthesis'
\usepackage{listings}
\lstdefinestyle{lstStyleBase}{%
  basicstyle=\small\ttfamily,
  aboveskip=\medskipamount,
  belowskip=\medskipamount,
  lineskip=0pt,
  boxpos=c,
  showlines=false,
  extendedchars=true,
  upquote=true,
  tabsize=2,
  showtabs=false,
  showspaces=false,
  showstringspaces=false,
  numbers=none,
  linewidth=\linewidth,
  xleftmargin=4pt,
  xrightmargin=0pt,
  resetmargins=false,
  breaklines=true,
  breakatwhitespace=false,
  breakindent=0pt,
  breakautoindent=true,
  columns=flexible,
  keepspaces=true,
  gobble=0,
  framesep=3pt,
  rulesep=1pt,
  framerule=1pt,
  backgroundcolor=\color{gray!5},
  stringstyle=\color{green!40!black!100},
  keywordstyle=\bfseries\color{blue!50!black},
  commentstyle=\slshape\color{black!60}}

\lstdefinestyle{lstStyleShell}{%
  style=lstStyleBase,
  frame=l,
  rulecolor=\color{shtred},
  language=bash}

\lstdefinestyle{lstStyleLaTeX}{%
  style=lstStyleBase,
  frame=l,
  rulecolor=\color{shtred},
  language=[LaTeX]TeX}

\lstnewenvironment{latex}{\lstset{style=lstStyleLaTeX}}{}
\lstnewenvironment{shell}{\lstset{style=lstStyleShell}}{}

\usepackage{hologo}
\ifthenelse{\equal{\currentfontset}{windows}}{
  \newfontface\EmojiFont{Segoe UI Emoji}[Renderer=HarfBuzz]
}{
  \ifthenelse{\equal{\currentfontset}{mac}}{
    \newfontface\EmojiFont{Apple Color Emoji}[Renderer=HarfBuzz]
  }{
    \newfontface\EmojiFont{Noto Color Emoji}[Renderer=HarfBuzz]
  }
}
\providecommand{\emoji}[1]{%
  \begingroup%
  \EmojiFont #1%
  \endgroup%
}
\IfFontExistsTF{Times New Roman}{
  \providefontfamily\timesnewroman{Times New Roman}
}{
  \providecommand{\timesnewroman}{\emph{Times New Roman missing}}
}
\IfFontExistsTF{Times}{
  \providefontfamily\timesfamily{Times}
}{
  \providecommand{\timesfamily}{\emph{Times missing}}
}
\IfFontExistsTF{XITS}{
  \providefontfamily\xitsfamily{XITS}
}{
  \providecommand{\xitsfamily}{\emph{XITS missing}}
}

\usepackage{subcaption}
\usepackage{ctable}

\begin{document}
\maketitle

\frontmatter
\begin{abstract}
  这篇文档阐述 \shtthesis 的使用方法,包括文档编译方式、文档类选项、文档类提供的功能及其他便于用户(\emph{也就是正在阅读这篇文档的您})快速撰写符合上海科技大学要求的学位论文的技术要点。

  \shtthesis 旨在以最简实现和最小依赖完整覆盖《上海科技大学研究生学位论文撰写规范(初稿)》的所有格式需求,且不为用户额外设限。从 v\version 起,仅需单个 \verb|shtthesis.cls| 文件即可满足排版需求。文档通过 \verb|\shtsetup| 命令统一设定学位论文信息,且仅提供满足格式需求的最少额外命令。用户可根据自身撰文习惯,引入额外的宏包和命令完成学位论文撰写。
\end{abstract}

\begin{abstract*}
  We elaborate on the usage of \shtthesis, which includes the compiling process, document class options, functionalities, and other additional technical points to facilitate the user (that is, \emph{you}) to quickly develop their thesis to meet the format requirements of ShanghaiTech University. 

  \shtthesis is intended to cover all requirements in 《上海科技大学研究生学位 论文撰写规范(初稿)》with minimal function set and introduce no restriction to the user. Beginning with v\version, the \verb|shtthesis.cls| file standalone is sufficient for the thesis typesetting. This document class uses \verb|\shtsetup| to uniformly set up thesis information and only provides minimum auxiliary commands to meet all format requirements. The user can import other packages and functionalities according to their writing habits. 
\end{abstract*}

\makeindices

\begin{nomenclatures}
  \header[单位]{符号}{说明}
  \item[${m^{2} \cdot s^{-2} \cdot K^{-1}}$]{$R$}{the gas constant}
  \item[${m^{2} \cdot s^{-2} \cdot K^{-1}}$]{$C_v$}{specific heat capacity at constant volume}
  \item[${m^{2} \cdot s^{-2} \cdot K^{-1}}$]{$C_p$}{specific heat capacity at constant pressure}
  \item[${m^{2} \cdot s^{-2}}$]{$E$}{specific total energy}
  \item[${kg \cdot m \cdot s^{-3} \cdot K^{-1}}$]{$k$}{thermal conductivity}
  \item[${kg \cdot m^{-1} \cdot s^{-2}}$]{$S_{ij}$}{deviatoric stress tensor}
  \item[${kg \cdot m^{-1} \cdot s^{-2}}$]{$\tau_{ij}$}{viscous stress tensor}
  \item[${1}$]{$\delta_{ij}$}{Kronecker tensor}
\end{nomenclatures}

\begin{nomenclatures}[缩写]
  \header{缩写}{全称}
  \item{CFD}{Computational Fluid Dynamics}
  \item{CFL}{Courant-Friedrichs-Lewy}
  \item{WENO}{Weighted Essentially Non-oscillatory}
  \item{ZND}{Zel'dovich-von Neumann-Doering}
\end{nomenclatures}

\begin{nomenclatures}[算子 \& 说明]
  \item{$\Delta$}{difference}
  \item{$\nabla$}{gradient operator}
  \item{$\delta^{\pm}$}{upwind-biased interpolation scheme}
\end{nomenclatures}

\mainmatter
\chapter{模板介绍}
\shtthesis (\textbf{S}hang\textbf{h}ai\textbf{T}ech University \textbf{THESIS}) 是根据《上海科技大学研究生学位论文撰写规范(初稿)》(下文简称《规范》)编写的、适用于上海科技大学学位论文写作的\emph{非官方} \LaTeX 模板。目前版本(v\version)提供了博士、硕士学位论文排版选项,且能够自动生成用于盲审的匿名版以及最终提交的非匿名版论文。

这篇文档将尽量详细地阐释 \shtthesis 文档类的使用方法和技巧。由于这篇文档直接使用 \shtthesis 排版,所以其源代码文件 \texttt{\jobname\-.tex} 也可以作为一个实际样例以供读者参考使用。

我们非常希望在 \shtthesis 后续版本中加入本科学位论文的排版选项,因此亟需有上海科技大学本科论文排版经验的同学参与到 \shtthesis 项目中。我们也计划将该使用说明和模板文件 \verb|shtthesis.cls| 统一重构为 \verb|.dtx| 文件。同时,我们也非常希望得到用户宝贵的反馈和建议。若您有意为 \shtthesis 贡献 Issues 和 Pull Requests,请移步至项目主页 \url{https://github.com/lirundong/sht-thesis}。

\chapter{模板使用}
\section{模板安装}
\shtthesis 提供的文档类型文件为 \verb|shtthesis.cls|,封面所需的上海科技大学校徽文件为 \verb|shanghaitech-logo.pdf|,用户仅需将这两个文件拷贝至自己的论文目录下即可完成模板安装。假设用户的论文文件为 \verb|thesis.tex|,则工作目录中必要的文件包括:
\begin{center}
  \begin{tabular}{ll}
    \toprule
    文件名称 & 说明 \\
    \midrule 
    \verb|shtthesis.cls| & 模板文档类型 \\
    \verb|shanghaitech-logo.pdf| & 上海科技大学校徽 \\
    \verb|thesis.tex| & 论文文件 \\
    \verb|reference.bib| & 参考文献默认文件名 \\
    \bottomrule
  \end{tabular}
\end{center}

\section{文档编译}
\shtthesis 支持使用 \hologo{XeLaTeX} 和 \hologo{LuaLaTeX} 编译(注意,\emph{不支持} \hologo{pdfLaTeX})。尝试编译本文档的源代码 \texttt{\jobname.tex} 至 PDF 文件是了解 \shtthesis 编译流程最直观的方法。我们推荐在最新的 \hologo{TeX} Live 环境下,使用 \verb|latexmk| 工具进行编译。Windows 及 Linux 用户请下载安装 \hologo{TeX} Live (\url{https://www.tug.org/texlive/}),macOS 用户请下载安装 Mac\hologo{TeX} (\url{https://www.tug.org/mactex/})。不推荐在 overleaf 等在线平台编译使用 \shtthesis,项目主页也不接受与 overleaf 相关的 Issues。

在完成环境配置后,即可使用 \verb|latexmk| 工具完成编译。使用 \hologo{XeLaTeX} 引擎进行编译的命令为:
\begin{shell}
> latexmk -pdfxe shtthesis-user-guide.tex
\end{shell}
若使用 \hologo{LuaLaTeX} 引擎,则编译命令为:
\begin{shell}
> latexmk -pdflua shtthesis-user-guide.tex
\end{shell}
一般来说,\hologo{XeLaTeX} 引擎的编译速度较快且占用资源较少,而 \hologo{LuaLaTeX} 引擎的编译结果似乎有更好的跨平台规范性。在 \shtthesis 开发过程中,曾出现过在 macOS 下 \hologo{XeLaTeX} 编译的 PDF 在 Windows 下无法打开,而 \hologo{LuaLaTeX} 编译结果正常的情况。

若使用 LuaHB\hologo{TeX} 引擎编译,还可进一步使用 emoji 等特殊功能。例如:\emoji{ 👴🏻👴🏼👴🏽👴🏾👴🏿👴👨🏻‍🦳🎅🏻🦹🏻‍♂️🕵🏼‍♂️👨🏼‍⚕️👨🏼‍🌾👨🏼‍🍳👨🏼‍💼👨🏼‍🔧👨🏼‍🔬👨🏼‍🎨👨🏼‍🚒👨🏼‍✈️👨🏼‍⚖️🧔🏼🤵🏼🧙🏻‍♂️🧖🏼‍♂️👨🏼‍🌾👮🏼‍♂️👨🏼‍🏫👨🏼‍🚀}。LuaHB\hologo{TeX} 引擎在 \hologo{TeX} Live 2019 下实现为 \verb|lualatex-dev|,在 \verb|latexmk| 的使用方法为:
\begin{shell}
> latexmk -pdflua -pdflualatex=lualatex-dev \
    shtthesis-user-guide.tex
\end{shell}

在编译用户自己的论文时,只需将上述命令中 \texttt{\jobname\-.tex} 替换为 \verb|thesis.tex| 即可。

\section{载入模板类}
模板安装完成后,在论文文件 \verb|thesis.tex| 开头使用
\begin{latex}
\documentclass{shtthesis}
\end{latex}
即可载入模板。\shtthesis 模板类的选项包括 \verb|anonymous|,以及传递给 \textsf{cexbook} 的其他选项。

\subsection{\texttt{anonymous} 选项}
为方便生成盲审所需的匿名版论文,在文档类传入 \verb|anonymous| 即可将论文 PDF 中英文封面的作者信息、导师信息,以及附录中的作者简历替换为匿名字符串,将作者论文发表、专利申请记录替换为匿名版本,并隐去文末的致谢部分:
\begin{latex}
\documentclass[anonymous]{shtthesis}
\end{latex}

\shtthesis 默认的匿名字符串为连续的三个英文星号 ***,该字符串可以在 \verb|\shtsetup| 中使用 anonymous-str 选项修改,详见第 TODO 节。论文文末的作者论文发表、专利申请记录分别使用非匿名的 \verb|publications|、\verb|patterns| 环境,及匿名的 \verb|publications*|、\verb|patterns*| 环境录入,并分别在非匿名版和匿名版论文中显示,详见第 TODO 节。

\subsection{传递给 \textsf{ctexbook} 的选项}
\shtthesis 实际使用 \CTeX 宏包提供的 \textsf{ctexbook} 文档类排版,故除 \verb|anonymous| 外的其余类参数会传递给 \textsf{ctexbook}。其中需要用户注意的选项为 fontset,即设定论文所用的字体集。\CTeX 宏包自身应该能够根据编译平台选择合适的字体集,然而我们在 \shtthesis 开发过程中发现,\CTeX 有时会将字体集回退至 \TeX Live 提供的最基础的 Fandol 字符集而造成缺字的情况。因此,我们推荐用户根据自身所用平台手动设置相应的 fontset。例如,在 Windows 平台下设置 \verb|fontset=windows|,在 macOS 平台下设置 \verb|fontset=mac|:
\begin{latex}
\documentclass[fontset=windows]{shtthesis}
\end{latex}

\shtthesis 主要支持的 \verb|fontset| 参数包括 \verb|windows|、\verb|mac| 和 \verb|fandol|,分别对应 Windows、Mac 和 Fandol 字符集。各字符集所用的字体见表~\ref{tab::fonts} 所示。

\begin{table}[htb]
  \centering
  \caption{不同字符集下 \shtthesis 所用字体}
  \label{tab::fonts}
  \begin{subtable}{\columnwidth}
    \centering
    \caption{\shtthesis 所用中文字体}
    \label{tab::chs_fonts}
    \begin{tabular}{l *{3}{c}}
      \toprule
       & Windows & Mac & Fandol \\
      \midrule
      宋体 & 中易宋体 & 华文宋体简体 & Fandol 宋体 \\
      黑体 & 中易黑体 & 华文黑体简体 & Fandol 黑体 \\
      楷书 & 中易楷体 & 华文楷体简体 & Fandol 楷体 \\
      仿宋 & 中易仿宋 & 华文仿宋简体 & Fandol 仿宋 \\
      \bottomrule
    \end{tabular}
  \end{subtable}
  \newline
  \vspace{12pt}
  \newline
  \begin{subtable}{\columnwidth}
    \centering
    \caption{\shtthesis 所用英文字体}
    \label{tab::eng_fonts}
    \begin{tabular}{l *{3}{c}}
      \toprule
       & Windows & Mac & Fandol \\
      \midrule
      Serif      & TeX Gyre Terms & TeX Gyre Terms & TeX Gyre Terms \\
      Sans Serif & TeX Gyre Heros & TeX Gyre Heros & TeX Gyre Heros \\
      Typewriter & TeX Gyre Cursor & TeX Gyre Cursor & TeX Gyre Cursor \\
      \bottomrule
    \end{tabular}
  \end{subtable}
\end{table}

需要注意 Fandol 字体集并不能完整覆盖 GBK,故对于部分生僻字可能出现漏字情况。\shtthesis 提供了 \verb|\currentfontset| 测试当前所用字符集,其用法为:
\begin{quotation}
当前文档所用字符集为 \texttt{\currentfontset}。
\end{quotation}
用户可以在不显著违反《规范》的前提下(例如,{\rmfamily TeX Gyre Terms}、{\xitsfamily XITS}、{\timesfamily Times} 和 {\timesnewroman Times New Roman} 几种字体几乎没有明显区别,见表~\ref{tab::serif_fonts})按照自身需求自定义字符集,或通过其他参数改变 \textsf{ctexbook} 的排版行为。\textsf{ctexbook} 可用参数详见 \CTeX 文档。\footnote{\url{http://mirrors.ctan.org/language/chinese/ctex/ctex.pdf}}

\begin{table}[htb]
  \centering
  \caption{不同英文衬线字体测试}
  \label{tab::serif_fonts}
  \begin{tabular}{ll}
    \toprule
    Font Name & Content \\
    \midrule
    TeX Gyre Terms & {\rmfamily The quick brown fox jumps over the lazy dog.} \\
    XITS & {\xitsfamily The quick brown fox jumps over the lazy dog.} \\
    Times & {\timesfamily The quick brown fox jumps over the lazy dog.} \\
    Times New Roman & {\timesnewroman The quick brown fox jumps over the lazy dog.} \\
    \bottomrule
  \end{tabular}
\end{table}

\section{设定论文信息}
\shtthesis 在 \verb|\shtsetup| 命令中,使用 key=value 方式统一设定论文信息。\shtthesis 需要知晓学位类型(v\version 支持 master 和 doctor)才能进行进一步排版;其次根据《规范》,博士学位论文和硕士学位论文需要在中英文封面中,依次列出论文密级、论文标题、作者姓名、导师信息、学位类别、一级学科、学校/学院名称及论文完成时间,并在中英文摘要之后列出论文关键词。以上信息均可通过 \verb|\shtsetup| 命令,在论文导言区(即 \verb|\documentclass| 之后、\verb|\begin{document}| 之前)设定。用户可以一次性调用 \verb|\shtsetup| 设定所有信息,也可分多次设定,但是最后一次设定一定要在 \verb|\begin{document}| 之前,并且需要注意 \verb|\shtsetup| 命令内\emph{不能有空行}。例如:
\begin{latex}
\shtsetup{
  degree = {master},
}
% other preamble contents...
\shtsetup{
  degree-name* = {Master~of~Science~in~Engineering},
}
\end{latex}

\makebiblio

\appendix

\backmatter
\begin{acknowledgement}
  最终致谢信息……
\end{acknowledgement}

\begin{resume}
  个人简历…… (仅非匿名环境显示)
\end{resume}

\begin{publications}
  论文发表…… (非匿名环境)
\end{publications}

\begin{publications*}
  论文发表…… (匿名环境)
\end{publications*}

\begin{patterns}
  专利申请或授权记录…… (非匿名环境)
\end{patterns}

\begin{patterns*}
  专利申请或授权记录…… (匿名环境)
\end{patterns*}

\begin{projects}
  个人参与的科研项目、获奖情况…… (仅非匿名环境显示)
\end{projects}

\end{document}
